\documentclass[a4paper]{article}

\usepackage[T1]{fontenc}
\usepackage[utf8]{inputenc}
\usepackage[english]{babel}


\usepackage[ruled,vlined,linesnumbered]{algorithm2e} %,noend


% ===== Graph =====
\usepackage{tikz}
% ===== Multicol =====
\usepackage{blindtext}
\usepackage{multicol}
% ===== Cancel =====
\usepackage{cancel}
% ===== Code =====
\usepackage{listings} 
\lstdefinestyle{mystyle}{
    breakatwhitespace=false,                   
    captionpos=b,                    
    keepspaces=true,                 
    numbers=left,                    
    numbersep=5pt,                  
    showspaces=false,                
    showstringspaces=false,
    showtabs=False,                  
    tabsize=2
}

\usepackage{hyperref}

\textwidth=450pt
\oddsidemargin=0pt
\textheight=665pt
\voffset=-50pt

\lstset{style=mystyle}

\usepackage{setspace}

\usepackage{amssymb}
\usepackage{amsthm}
\usepackage{mathtools}
\usepackage{bm}

\singlespacing
% ===================
\mathtoolsset{showonlyrefs}  
\hypersetup{
    colorlinks=true,
    linkcolor=black,
    filecolor=black,      
    urlcolor=black,
}

\AtBeginDocument{\renewcommand\proofname{Proof}}

\newcommand{\pluseq}{\mathrel{{+}{=}}}

\newtheorem{theorem}{Theorem}
\newtheorem{corollary}{Corollary}
\newtheorem{lemma}{Lemma}
\newtheorem{remark}{Remark}
\newtheorem{definition}{Definition}

\setcounter{secnumdepth}{3}
\setcounter{tocdepth}{3}

% \title{Statistical Methods for Maching Learning}
% \author{Davide Cologni}
% \date{}

\makeindex

\begin{document}

% \maketitle


\begin{titlepage}
    \begin{center}
        \LARGE
        % \text{University of Milan}

        % \vspace*{1cm}

        \textbf{MSc in Computer Science} \\
        at University of Milan

        \vspace*{1cm}

        
        \huge
        Statistical Methods for Machine Learning\\
        Kernelized Linear Predictor\\
        
        \large course held by \textbf{Nicoló Cesa-Bianchi}
        

        \normalsize
        \vspace*{4cm}

        \begin{minipage}[t]{0.47\textwidth}
	       {Email: } \vspace{0.3em} \\
              {\large \href{davide.cologni@studenti.unimi.it}{davide.cologni@studenti.unimi.it}} \vspace{1em}  \\
        \end{minipage}
        \hfill
        \begin{minipage}[t]{0.47\textwidth}\raggedleft
	       {Created by:} \hspace{-0.9em} \vspace{0.3em} \\
              {\large \textbf{Davide Cologni}} \\
              {\footnotesize mat. 09732A}
        \end{minipage}

        \vfill
        Academic year of 2023/2024
            
    \end{center}
\end{titlepage}

\clearpage\null\newpage

\newpage
% \setlength{\parskip}{0.15em}
\tableofcontents
\setlength{\parindent}{0pt}
\setlength{\parskip}{0.8em}


\clearpage\null\newpage


\newpage
\newpage
\section{introduction}
% dataset analysis and preprocessing
%   - outliers removal
%   - feature correlation
%   - feature expansion
%     to being able to express non-homogeneous linear separator (separator that are not in the origin) we add a constant feature of one to the training set in this way... (formula) 
%   - standardization (also advantage from the theorical prospective) (avoi data leakage)
% Perceptron
%   - naive
%     - results
%   - with Feature expansion (degree 2)
%     - results
% Support Vector Machine (Pegasos)
%   - naive
%       - Hyperparameter tuning
%          grid search with his implementation
%       - implementation details
%           (compare the implementation with the one on the paper)
%           typo: the index i_t in the summation
%       - results
%           - Overfitting and under fitting (theoric view)
%   - logistic regression
%       - results
%           - Overfitting and under fitting (theoric view)
%   - with Feature expansion
%   - logistic regression with Feature expansion
% Kernels


\end{document}