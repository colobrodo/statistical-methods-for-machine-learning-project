\newpage
\section{introduction}
\subsection{Project Description}
The goal of this project is to train different learning algorithms to solve the binary classification problem.\\
Specifically, we need to predict the value of the label in the $y$ column based on the numerical features $x1$ through $x10$.\\
As mentioned in the description of the project, I used the zero-one loss as a metric to evaluate the performance of the algorithms.\\
In the following chapter I describe how I analyzed and preprocessed the dataset to improve the performance of the modules.\\
In the following chapters I describe how I implemented each algorithm and how I chose the hyperparameters for them.\\
I have also analyzed how varying the hyperparameter affects the training and testing error and explained these empirical results with the theoretical background provided by the lectures.\\

I implemented the following algorithms:
\begin{itemize}
\item Perceptron
\item Pegasos
\item Logistic Pegasos
\item Feature expanded Perceptron (with 2nd degree polynomial expansion)
\item Feature expanded Pegasos (with 2nd degree polynomial expansion)
\item Feature expanded Pegasos with logistic loss (with 2nd degree polynomial expansion)
\item Kernel Perceptron
\item Kernel Pegasos
\end{itemize}

\subsection{Project Structure}
The project is divided into the following folders:
\begin{itemize}
\item datasets: contains the provided dataset
\item models: a set of pre-trained models, created using the train subcommand
\item src: the source code of the project, the entry point is main.py and can be used both to train the models from the dataset and to run them
\item report: this folder contains the source for this report
\end{itemize}

\subsection{Usage}
The entry point to the project is the {\bf main.py} file. It can be called with the command line argument and provides two subcommands 'train' and 'run'
The first subcommand requires the name of the algorithm to be trained and stores a predictor in the path provided in the output argument, serialized using the Python pickle module.

\begin{lstlisting}[language=bash]
  $ python src/main.py train pegasos models/pegasos.pkl
\end{lstlisting}

% TODO: output example

The run subcommand takes a serialized model and then prints its training and test errors

\begin{lstlisting}[language=bash]
  $ python src/main.py run models/pegasos.pkl
\end{lstlisting}

% TODO: output example

There are other options available that are described using the '--help' option, they will be described in the next sections of this report as they come up.