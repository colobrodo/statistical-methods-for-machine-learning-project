\newpage

\section{Conclusion}
I started this report by describing how to use the project and then moved on the preprocessing the data: which methods are used and based on which observation.\\
Then I implemented the Perceptron, Pegasos and logistic regression algorithms, also with second degree feature expansion and kernelized variants (the latter only for Perceptron and Pegasos).\\
I have used the lecture notes to introduce the theoretical background for understanding how these algorithms work at the beginning of each chapter, then explained the experimental data based on that.\\
I have discussed the implementation details and how I choose the hyperparameters for each algorihtm.\\
I have also identified for which hyperparameters the algorithm overfits and underfits, explained using the theoretical prospective.\\
The best performing algorithm in all variants is the Pegasos algorithm.\\
We also observed that the feature-extended version performs better than the naive one, but the best results come from implementing the algorithm in combination with kernels.\\
More specifically, the third degree polynomial kernel is the best performing kernel for both Perceptron and Pegasos, intuitively because the algorithm is able to express a higher degree polynomial curve in the original space.\\
The feature expansion is of second degree, so it has less expressive power than the chosen kernel.\\
But increasing the degree of the polynomial curve doesn't necessarily mean finding a better predictor: we also see that for $n = 4$ the curve overfits the data, giving a low training error but a high test error.\\
I also observed Gaussian kernels are not effective as the polynomial one because they tend to overfit more easily.\\
